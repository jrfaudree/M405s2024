% !TEX TS-program = pdflatexmk
\documentclass[12pt]{article}

% Layout.
\usepackage[top=.75in, bottom=0.75in, left=.75in, right=.75in, headheight=1in, headsep=6pt]{geometry}

% Fonts.
\usepackage{mathptmx}
\usepackage[scaled=0.86]{helvet}
\renewcommand{\emph}[1]{\textsf{\textbf{#1}}}

% Misc packages.
\usepackage{amsmath,amssymb,latexsym}
\usepackage{graphicx,tikz}
\usepackage{array}
\usepackage{xcolor}
\usepackage{multicol}
\usepackage{tabularx,colortbl}
\usepackage{enumitem}
%to make tikz pics work
\usepackage{tikz,pgfplots}
\usetikzlibrary{arrows}
\newcommand{\midarrow}{\tikz \draw[-triangle 90] (0,0) -- +(.1,0);}

\usepackage[colorlinks=true]{hyperref}

% Paragraph spacing
\parindent 0pt
\parskip 6pt plus 1pt
\def\tableindent{\hskip 0.5 in}
\def\ts{\hskip 1.5 em}

\usepackage{fancyhdr}
\pagestyle{fancy} 
\lhead{\large\sf\textbf{MATH 405}}
\rhead{\large\sf\textbf{Spring 2024}}
\chead{\large\sf\textbf{HW 2 }}

\newcommand{\localhead}[1]{\par\smallskip\textbf{#1}\nobreak\\}%
\def\heading#1{\localhead{\large\emph{#1}}}
\def\subheading#1{\localhead{\emph{#1}}}

%% Special Math Symbol shortcuts
\newcommand{\bbN}{\mathbb{N}}
\newcommand{\bbZ}{\mathbb{Z}}
\newcommand{\bbR}{\mathbb{R}}
\newcommand{\rad}{\text{rad}}
\newcommand{\diam}{\text{diam}}

\usetikzlibrary{calc,arrows.meta}
\usetikzlibrary{arrows}
\newcommand{\marrow}{\tikz \draw[-triangle 90] (0,0) -- +(.1,0);}


\begin{document}

\textbf{Observe} that the document has spaced out the problems so that there space for comments.\\

Chapter 2 Problems\\

\begin{enumerate}
\item Use Proof by Induction to prove the statement below.\\

\begin{quote} For all $n \in \bbN,$ $\displaystyle \frac{1}{2}+\frac{1}{6}+ \cdots + \frac{1}{n(n+1)}=\frac{n}{n+1}.$ \end{quote}

\textbf{Proof:} Your proof goes here.

\newpage

\item For $a=23771$ and $b=19945$, calculate $gcd(a,b)$ and find integers $s$ and $t$ such that 
$$as+bt=gcd(a,b).$$

\textbf{Answer:} Your answer goes here.

\newpage

\item Suppose that $a$ and $b$ are integers such that $gcd(a,b)=1.$ Let $s$ and $t$ be integers such that $as+bt=1.$ Prove that $gcd(a,s)=gcd(r,b)=gcd(s,t)=1.$\\

\textbf{Proof:} Your proof goes here.

\newpage

\item Let $a,b,c \in \bbN.$ Prove that if $gcd(a,b)=1$ and $a \: | \: bc,$ then $a \: | \: c.$\\

\textbf{Proof:} Your proof goes here.

\newpage

Chapter 3 Problems

\item This question is about the symmetries of a \emph{square}, as opposed to the example of the rectangle at the beginning of this section. You may want to use a drawing or permutations to describe your ideas but you don't have to. It's OK to have essay-style answers.\\

\begin{tikzpicture}[scale=1.5]
\draw (0,0) -- (1,0) -- (1,1) -- (0,1) -- (0,0);
\node at (-.25,-.25){$A$};
\node at (-.25,1.25){$B$};
\node at (1.25,1.25){$C$};
\node at (1.25,-.25){$D$};
\end{tikzpicture}

	\begin{enumerate}
	\item Describe the symmetries of the square.\\
	\textbf{Answer:} Your answer goes here.
	\vfill
	\item  How many symmetries of the square are there?\\
	\textbf{Answer:} Your answer goes here. It's going to be just a number!
	\vfill
	\item  How many permutations of the set $\{A,B,C,D\}$ are there?\\
	\textbf{Answer:} Your answer goes here. It's going to be just a number! A little explanation of your reasoning here never hurts!
	\vfill
	\item  Give an example of a permutation of the set $\{A,B,C,D\}$ that cannot correspond to a symmetry of the square pictured above?\\
	\textbf{Answer:} Your answer goes here. Justify your answer
		\vfill
	\end{enumerate}
\newpage	
\item Let $S=\bbR \backslash \{-1\}$ and define a binary operation $*$ by $a * b =a+b+ab.$ Prove that $(S,*)$ is an abelian group.\\

\textbf{Proof:} Your proof goes here.

\newpage

\item Give an example of two elements $A$ and $B$ in $GL_2(\bbR)$ with $AB \not = BA.$ \\

\textbf{Answer:} Your answer goes here.

\newpage

\item Given the groups $\bbR^*$ and $\bbZ$, let $G=\bbR^* \times \bbZ.$ Define a binary operation $\circ$ on $G$ by $(a,m) \circ (b,n)=(ab,m+n).$ Prove that $(G,\circ)$ is a group.

\textbf{Proof:} Your proof goes here.

\newpage

\item Let $(G,\circ)$ be a group. Let $g_1,g_2, \cdots, g_n \in G.$ Show that the inverse of $g_1g_2\cdots g_n$ is $g_n^{-1}\cdots g_2^{-1}g_1^{-1}.$\\

\textbf{Proof:} Your proof goes here.


\newpage


\end{enumerate}
\end{document}