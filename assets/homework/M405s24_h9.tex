% !TEX TS-program = pdflatexmk
\documentclass[12pt]{article}

% Layout.
\usepackage[top=.75in, bottom=0.75in, left=.75in, right=.75in, headheight=1in, headsep=6pt]{geometry}

% Fonts.
\usepackage{mathptmx}
\usepackage[scaled=0.86]{helvet}
\renewcommand{\emph}[1]{\textsf{\textbf{#1}}}

% Misc packages.
\usepackage{amsmath,amssymb,latexsym}
\usepackage{graphicx,tikz}
\usepackage{array}
\usepackage{xcolor}
\usepackage{multicol}
\usepackage{tabularx,colortbl}
\usepackage{enumitem}
%to make tikz pics work
\usepackage{tikz,pgfplots}
\usetikzlibrary{arrows}
\newcommand{\midarrow}{\tikz \draw[-triangle 90] (0,0) -- +(.1,0);}

\usepackage[colorlinks=true]{hyperref}

% Paragraph spacing
\parindent 0pt
\parskip 6pt plus 1pt
\def\tableindent{\hskip 0.5 in}
\def\ts{\hskip 1.5 em}

\usepackage{fancyhdr}
\pagestyle{fancy} 
\lhead{\large\sf\textbf{MATH 405}}
\rhead{\large\sf\textbf{Spring 2024}}
\chead{\large\sf\textbf{HW 9}}

\newcommand{\localhead}[1]{\par\smallskip\textbf{#1}\nobreak\\}%
\def\heading#1{\localhead{\large\emph{#1}}}
\def\subheading#1{\localhead{\emph{#1}}}

%% Special Math Symbol shortcuts
\newcommand{\bbN}{\mathbb{N}}
\newcommand{\bbZ}{\mathbb{Z}}
\newcommand{\bbR}{\mathbb{R}}
\newcommand{\bbQ}{\mathbb{Q}}

\newcommand{\rad}{\text{rad}}
\newcommand{\diam}{\text{diam}}

\usetikzlibrary{calc,arrows.meta}
\usetikzlibrary{arrows}
\newcommand{\marrow}{\tikz \draw[-triangle 90] (0,0) -- +(.1,0);}


\begin{document}
\begin{enumerate}
\item Let $\psi: G \to H$ be a group homomorphism. Prove that $\psi$ is one-to-one if an only if $\psi^{-1}(e_H)=e_G.$\\ \textbf{Hint:} You should use the First Isomorphism Theorem to prove one direction of the if and only if statement.\\

\textbf{Proof:}\\

\vfill

\item Prove that if $G$ is a finite group and $\phi: G \to H$ is a group homomorphism, then $|\phi(G)|$ divides both $|G|$ and $|H|.$\\

\textbf{Proof:}\\

\vfill
\newpage
\item Find all of the homomorphisms from $\phi: \mathbb{Z} \to \mathbb{Z}.$ Justify your answer.\\
\textbf{Hint:} For this problem and the next, you may want to use some facts you proved in HW 8.\\

\textbf{Answer:} \\
\textbf{Justification:} \\
\vfill

\item 
	\begin{enumerate}
	\item Find all of the automorphisms of $\mathbb{Z}_8$ and explain your reasoning. \\

\textbf{Answer with explanation:}\\

	\item Prove that $\text{Aut}(\mathbb{Z}_8)\cong U(8).$ Recall that $\text{Aut}(G)$ is the group of automorphisms of $G$ under the operation of function composition.\\

\textbf{Proof:}\\
	\end{enumerate}

\vfill

\item 
	\begin{enumerate}
	\item Find all abelian groups of order $n$ for $n \in \{15,16,17,18,19,20\}.$\\

	\textbf{Answer:}\\

	\vfill
	\item For each nonisomorphic group of order $18$, find an element of order 3.\\
	
	\textbf{Answer:}\\

	\vfill
	\end{enumerate}
\newpage
\item Which of the following sets are rings? If it is a ring, does it have a multiplicative identity? Is it commutative? An integral domain? A division ring? What are its units, if any? Is it a field? (You decide your explanations. Speak to your future self!)\\
	\begin{enumerate}
	\item $7\mathbb{Z}$ with usual addition and multiplication\\
	
	\begin{tabular}{lr}
	question&answer\\
	\hline
	\textbf{a ring?}& Answer here!\\
	\textbf{with unity?}& Answer here!\\
	\textbf{commutative?}& Answer here!\\
	\textbf{an integral domain?}& Answer here!\\
	\textbf{its units?}& Answer here!\\
	\textbf{a division ring?}& Answer here!\\
	\textbf{a field?}& Answer here!\\
	\end{tabular}
	\vfill
	
	\item $\mathbb{Z}_7$ with usual addition and multiplication\\
	
	\begin{tabular}{lr}
	question&answer\\
	\hline
	\textbf{a ring?}&\\
	\textbf{with unity?}&\\
	\textbf{commutative?}&\\
	\textbf{an integral domain?}&\\
	\textbf{its units?}&\\
	\textbf{a division ring?}&\\
	\textbf{a field?}&\\
	\end{tabular}

	
	\vfill
	
	\item $\mathbb{Z}_{18}$ with usual addition and multiplication\\
	
	\begin{tabular}{lr}
	question&answer\\
	\hline
	\textbf{a ring?}&\\
	\textbf{with unity?}&\\
	\textbf{commutative?}&\\
	\textbf{an integral domain?}&\\
	\textbf{its units?}&\\
	\textbf{a division ring?}&\\
	\textbf{a field?}&\\
	\end{tabular}

	
	\vfill

	\item $\mathbb{Q}(\sqrt{2})=\{ a+b\sqrt{2} \, :\, a,b \in \mathbb{Q}\}$ with usual addition and multiplication\\
	
	\begin{tabular}{lr}
	question&answer\\
	\hline
	\textbf{a ring?}&\\
	\textbf{with unity?}&\\
	\textbf{commutative?}&\\
	\textbf{an integral domain?}&\\
	\textbf{its units?}&\\
	\textbf{a division ring?}&\\
	\textbf{a field?}&\\
	\end{tabular}
	
	\vfill
	\item $R=\{ a+b\sqrt[3]{2} \, :\, a,b \in \mathbb{Q}\}$ with usual addition and multiplication\\
	
	\begin{tabular}{lr}
	question&answer\\
	\hline
	\textbf{a ring?}&\\
	\textbf{with unity?}&\\
	\textbf{commutative?}&\\
	\textbf{an integral domain?}&\\
	\textbf{its units?}&\\
	\textbf{a division ring?}&\\
	\textbf{a field?}&\\
	\end{tabular}

	
	\vfill
	
	
	\item $\mathbb{M}_2({\mathbb{Z}_{2}}),$ the set of $2 \times 2$ matrices with entries from $\mathbb{Z}_2$ with usual matrix addition and multiplication\\
	
	\begin{tabular}{lr}
	question&answer\\
	\hline
	\textbf{a ring?}&\\
	\textbf{with unity?}&\\
	\textbf{commutative?}&\\
	\textbf{an integral domain?}&\\
	\textbf{its units?}&\\
	\textbf{a division ring?}&\\
	\textbf{a field?}&\\
	\end{tabular}

	
	\vfill
	\item $\mathbb{R}_1[x]=\{ax+b \, : \, a,b \in \mathbb{R}\},$ the set of linear polynomials with real coefficients with the usual addition and multiplication\\
	
	\begin{tabular}{lr}
	question&answer\\
	\hline
	\textbf{a ring?}&\\
	\textbf{with unity?}&\\
	\textbf{commutative?}&\\
	\textbf{an integral domain?}&\\
	\textbf{its units?}&\\
	\textbf{a division ring?}&\\
	\textbf{a field?}&\\
	\end{tabular}

	
	\vfill

	\item $\mathbb{R}_\infty[x]$ the set of all polynomials with real coefficients with the usual addition and multiplication\\
	
	\begin{tabular}{lr}
	question&answer\\
	\hline
	\textbf{a ring?}&\\
	\textbf{with unity?}&\\
	\textbf{commutative?}&\\
	\textbf{an integral domain?}&\\
	\textbf{its units?}&\\
	\textbf{a division ring?}&\\
	\textbf{a field?}&\\
	\end{tabular}

	
	\vfill
	\end{enumerate}
	
	
\end{enumerate}
\end{document}

	
	
	
	
	
	

 

