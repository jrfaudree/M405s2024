% !TEX TS-program = pdflatexmk
\documentclass[12pt]{article}

% Layout.
\usepackage[top=.75in, bottom=0.75in, left=.75in, right=.75in, headheight=1in, headsep=6pt]{geometry}

% Fonts.
\usepackage{mathptmx}
\usepackage[scaled=0.86]{helvet}
\renewcommand{\emph}[1]{\textsf{\textbf{#1}}}

% Misc packages.
\usepackage{amsmath,amssymb,latexsym}
\usepackage{graphicx,tikz}
\usepackage{array}
\usepackage{xcolor}
\usepackage{multicol}
\usepackage{tabularx,colortbl}
\usepackage{enumitem}
%to make tikz pics work
\usepackage{tikz,pgfplots}
\usetikzlibrary{arrows}
\newcommand{\midarrow}{\tikz \draw[-triangle 90] (0,0) -- +(.1,0);}

\usepackage[colorlinks=true]{hyperref}

% Paragraph spacing
\parindent 0pt
\parskip 6pt plus 1pt
\def\tableindent{\hskip 0.5 in}
\def\ts{\hskip 1.5 em}

\usepackage{fancyhdr}
\pagestyle{fancy} 
\lhead{\large\sf\textbf{MATH 405}}
\rhead{\large\sf\textbf{Spring 2024}}
\chead{\large\sf\textbf{HW 8}}

\newcommand{\localhead}[1]{\par\smallskip\textbf{#1}\nobreak\\}%
\def\heading#1{\localhead{\large\emph{#1}}}
\def\subheading#1{\localhead{\emph{#1}}}

%% Special Math Symbol shortcuts
\newcommand{\bbN}{\mathbb{N}}
\newcommand{\bbZ}{\mathbb{Z}}
\newcommand{\bbR}{\mathbb{R}}
\newcommand{\bbQ}{\mathbb{Q}}

\newcommand{\rad}{\text{rad}}
\newcommand{\diam}{\text{diam}}

\usetikzlibrary{calc,arrows.meta}
\usetikzlibrary{arrows}
\newcommand{\marrow}{\tikz \draw[-triangle 90] (0,0) -- +(.1,0);}


\begin{document}

Problems 1 and 2 below use a problem from HW 6 \#9 about \emph{automorphisms} of groups, restated below. You may want to reference the \text{Fact} in your proofs.\\

\textbf{Definition 1:} Let $G$ be a group. An isomorphism $\phi: G \to G$ is called an \emph{automorphism}. (That is, an automorphism is an isomorphism from a group to itself.)\\

\textbf{Fact (that you proved):} Let $G$ be a group and $g \in G.$ The function $f_g: G \to  G$ defined as 
$$f_g(x)=gxg^{-1}$$ is an automorphism of $G.$\\ 

\textbf{Definition 2:} Let $G$ be a group. An {automorphism} $f:G \to G$ defined by $f_g(x)=gxg^{-1}$ is called an \emph{inner automorphism.} \\

\begin{enumerate}
%%%PROBLEM 1
\item 
	\begin{enumerate}
	\item Prove that if $G$ is a group with subgroup $H$, then the set $gHg^{-1}=\{ghg^{-1} \, : \, h \in H\}$ is a subgroup of $G.$\\
	
	\textbf{Proof:}
	\vfill
	
	\item Prove that if a group $G$ has exactly one subgroup $H$ of order $k,$ than $H$ must be normal in $G.$\\
	
	\textbf{Proof:}
	\vfill
	\end{enumerate}
\newpage
%%%%PROBLEM 2
\item 
	\begin{enumerate}
	\item Let $G=S_3$ and $g=(12).$ Describe the inner automorphism $f_g$ by filling out the table below. (Note that I filled out one row for you.)\\
	
	\begin{tabular}{c|c}
	$x$ & $f_g(x)$ \\
	\hline
	$()$ & $(12)()(12)=()$ \\
	$(12)$ & \\
	$(13)$ & \\
	$(23)$ & \\
	$(123)$ & \\
	$(132)$ & \\
	\end{tabular}
	
	\item Let $G=\bbZ_3$ and $g=1.$ Describe the inner automorphism $f_g$ by filling out the table below. \\
	
	\begin{tabular}{c|c}
	$x$ & $f_g(x)$ \\
	\hline
	$0$ & \\
	$1$ & \\
	$2$ & \\
	\end{tabular}
		
	\item If $G$ is abelian, what can you conclude about inner automorphisms of $G$? Justify your answer.\\
	
	\textbf{Answer:}
	\vfill
	
	\item Let $G=\bbZ_3.$ Describe an automorphism of $G$ that is not an inner automorphism.\\
	
	\begin{tabular}{c|c}
	$x$ & $f_g(x)$ \\
	\hline
	$0$ & \\
	$1$ & \\
	$2$ & \\
	\end{tabular}
	
	\item You have shown that some automorphisms can be constructed as inner automorphisms, but not all are of that form. Let $Aut(G)$ be the set of all automorphisms of the group $G.$ Prove that this set forms a group under the operation of function composition. (That is, you are proving that $Aut(G) \leq S_G.$)\\
	
	\textbf{Proof:}
	\vfill
	
	\end{enumerate}
\newpage
\item Let the function $f: \bbZ_8 \to \bbZ_{20}$ be defined at $f(n)=5n.$ Prove that $f$ is a homomorphism and determine its kernel and its image.\\

\textbf{Proof:}  \\
	
\textbf{kernel:}\\

\textbf{image:}\\

\newpage
\item For each map below, determine if it is a homomorphism. (You don't have to \emph{prove} it is or isn't a homomorphism.) If it is a homomorphism, determine its kernel and its image.
	\begin{enumerate}
	\item $\phi: \bbR^* \to GL_2(\bbR)$ defined by $\phi(a)= \begin{pmatrix} 1&0\\0&a \end{pmatrix}.$\\
	\textbf{Answer:}
	\vfill
	\item $\phi: \bbR \to GL_2(\bbR)$ defined by $\phi(a)= \begin{pmatrix} 1&0\\a&1 \end{pmatrix}.$\\
	\textbf{Answer:}
	\vfill
	\item $\phi: GL_2(\bbR) \to  \bbR $ defined by $\phi\left( \, \begin{pmatrix} a&b\\c&d \end{pmatrix} \,\right) =a+d$\\
	\textbf{Answer:}
	\vfill
	\item $\phi: GL_2(\bbR) \to  \bbR^* $ defined by $\phi\left( \, \begin{pmatrix} a&b\\c&d \end{pmatrix} \,\right) =ad-bc$\\
	\textbf{Answer:}
	\vfill
	\item $\phi: \mathbb{M}_2(\bbR) \to  \bbR$ defined by $\phi\left( \, \begin{pmatrix} a&b\\c&d \end{pmatrix} \,\right) =b$ where $\mathbb{M}_2(\bbR)$ is the additive group of $2 \times 2$ matrices with entries in $\bbR.$\\
	\textbf{Answer:}
	\vfill
	\end{enumerate}
\newpage
\item Let $A$ be an $m \times n$ matrix. Show that matrix multiplication, $x \to Ax,$ defines a homomorphism $\phi: \bbR^n \mapsto \bbR^m.$\\

\textbf{Proof:}
	\vfill
	
\item If $G$ is an abelian group and $n \in \bbN$, show that $\phi: G \to G$ defined by $g \mapsto g^n$ is a group homomorphism.\\

\textbf{Proof:}
	\vfill
\newpage
\item Show that a homomorphism defined on a cyclic group is completely determine by its action on the generator of the group. \\

\textbf{Proof:}
	\vfill
\item If $H$ and $K$ are normal subgroups of $G$ and $H \cap K=\{e\},$ prove that $G$ is isomorphic to a subgroup of $G/H \times G/K.$\\

\textbf{Proof:}
	\vfill

\end{enumerate}
\end{document}




