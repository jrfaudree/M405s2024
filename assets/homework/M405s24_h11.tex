% !TEX TS-program = pdflatexmk
\documentclass[12pt]{article}

% Layout.
\usepackage[top=.75in, bottom=0.75in, left=.75in, right=.75in, headheight=1in, headsep=6pt]{geometry}

% Fonts.
\usepackage{mathptmx}
\usepackage[scaled=0.86]{helvet}
\renewcommand{\emph}[1]{\textsf{\textbf{#1}}}

% Misc packages.
\usepackage{amsmath,amssymb,latexsym}
\usepackage{graphicx,tikz}
\usepackage{array}
\usepackage{xcolor}
\usepackage{multicol}
\usepackage{tabularx,colortbl}
\usepackage{enumitem}
%to make tikz pics work
\usepackage{tikz,pgfplots}
\usetikzlibrary{arrows}
\newcommand{\midarrow}{\tikz \draw[-triangle 90] (0,0) -- +(.1,0);}

\usepackage[colorlinks=true]{hyperref}

% Paragraph spacing
\parindent 0pt
\parskip 6pt plus 1pt
\def\tableindent{\hskip 0.5 in}
\def\ts{\hskip 1.5 em}

\usepackage{fancyhdr}
\pagestyle{fancy} 
\lhead{\large\sf\textbf{MATH 405}}
\rhead{\large\sf\textbf{Spring 2024}}
\chead{\large\sf\textbf{HW 11}}

\newcommand{\localhead}[1]{\par\smallskip\textbf{#1}\nobreak\\}%
\def\heading#1{\localhead{\large\emph{#1}}}
\def\subheading#1{\localhead{\emph{#1}}}

%% Special Math Symbol shortcuts
\newcommand{\bbN}{\mathbb{N}}
\newcommand{\bbZ}{\mathbb{Z}}
\newcommand{\bbR}{\mathbb{R}}
\newcommand{\bbQ}{\mathbb{Q}}
\newcommand{\bbC}{\mathbb{C}}
\newcommand{\bbM}{\mathbb{M}}


\newcommand{\rad}{\text{rad}}
\newcommand{\diam}{\text{diam}}

\usetikzlibrary{calc,arrows.meta}
\usetikzlibrary{arrows}
\newcommand{\marrow}{\tikz \draw[-triangle 90] (0,0) -- +(.1,0);}


\begin{document}
\begin{enumerate}
\item List all distinct polynomials of degree 3 or less in $\bbZ_2.$\\

\item Compute each of the following if $p(x)=5x^2+3x-4$ and $q(x)=4x^2-x+9$ are polynomials in $\bbZ_{12}.$\\
	\begin{enumerate}
	\item $p(x)+q(x)$
	\item $p(x)\cdot q(x)$
	\item $(p(x))^2$
	\end{enumerate}
\item Use the Division Algorithm to find $q(x)$ and $r(x)$ such that $a(x)=q(x)b(x)+r(x)$ where $\text{deg}(r(x)) < \text{deg}(b(x))$ or $r(x)$ is the zero polynomial.\\
For this problem, it is OK to just state your $q(x)$ and $r(x)$, but you want to make sure you know how to find them. To format the division algorithm in \LaTeX \, is unnecessarily tedious. 
	\begin{enumerate}
	\item $a(x)= 5x^3+6x^2-3x+4,$ $b(x)=x-2$,  in the polynomial ring $\bbZ_7[x]$\\
	\item $a(x)= x^5+x^3-x^2-x,$ $b(x)=x^3+x$,  in the polynomial ring $\bbZ_2[x]$\\
	\end{enumerate}
\item Final all zeros for each of the following polynomials or demonstrate that none exist.\\
	\begin{enumerate}
	\item $5x^3+4x^2-x+9$ in $\bbZ_{12}[x]$ (Hint: There is an easier way than just testing all 12 possibilities!\\
	\item $5x^4+2x^2-3$ in $\bbZ_7[x]$\\
	\item $x^5+x^3+1$ in $\bbZ_2[x]$\\
	\end{enumerate}
	
\item Find a polynomial $p(x)$ in $\bbZ_4$ of degree at least 2 such that $p(x)$ is a unit.\\

\item Give two different factorizations of $x^2+x+8$ in $\bbZ_{10}[x].$\\

\item 
	\begin{enumerate}
	\item Give an example of polynomials $a(x)$ and $b(x)$ in $\bbZ[x]$ such that the Division Algorithm fails.\\
	\item Explain what hypothesis in Theorem 17.6 fails to hold for the polynomials and/or the ring from part (a).\\
	\end{enumerate} 
\end{enumerate}
\end{document}

	
	
	
	
	
	

 

