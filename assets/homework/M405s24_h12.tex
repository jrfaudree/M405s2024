% !TEX TS-program = pdflatexmk
\documentclass[12pt]{article}

% Layout.
\usepackage[top=.75in, bottom=0.75in, left=.75in, right=.75in, headheight=1in, headsep=6pt]{geometry}

% Fonts.
\usepackage{mathptmx}
\usepackage[scaled=0.86]{helvet}
\renewcommand{\emph}[1]{\textsf{\textbf{#1}}}

% Misc packages.
\usepackage{amsmath,amssymb,latexsym}
\usepackage{graphicx,tikz}
\usepackage{array}
\usepackage{xcolor}
\usepackage{multicol}
\usepackage{tabularx,colortbl}
\usepackage{enumitem}
%to make tikz pics work
\usepackage{tikz,pgfplots}
\usetikzlibrary{arrows}
\newcommand{\midarrow}{\tikz \draw[-triangle 90] (0,0) -- +(.1,0);}

\usepackage[colorlinks=true]{hyperref}

% Paragraph spacing
\parindent 0pt
\parskip 6pt plus 1pt
\def\tableindent{\hskip 0.5 in}
\def\ts{\hskip 1.5 em}

\usepackage{fancyhdr}
\pagestyle{fancy} 
\lhead{\large\sf\textbf{MATH 405}}
\rhead{\large\sf\textbf{Spring 2024}}
\chead{\large\sf\textbf{HW 12}}

\newcommand{\localhead}[1]{\par\smallskip\textbf{#1}\nobreak\\}%
\def\heading#1{\localhead{\large\emph{#1}}}
\def\subheading#1{\localhead{\emph{#1}}}

%% Special Math Symbol shortcuts
\newcommand{\bbN}{\mathbb{N}}
\newcommand{\bbZ}{\mathbb{Z}}
\newcommand{\bbR}{\mathbb{R}}
\newcommand{\bbQ}{\mathbb{Q}}
\newcommand{\bbC}{\mathbb{C}}
\newcommand{\bbM}{\mathbb{M}}


\newcommand{\rad}{\text{rad}}
\newcommand{\diam}{\text{diam}}

\usetikzlibrary{calc,arrows.meta}
\usetikzlibrary{arrows}
\newcommand{\marrow}{\tikz \draw[-triangle 90] (0,0) -- +(.1,0);}


\begin{document}
\begin{enumerate}
\item For every polynomial of degree 2or 3 in $\bbZ_2$ (see the table below), either factor it completely or state that it is irreducible. \\
\begin{center}
\begin{tabular}{lll}
polynomial & factored form&zeros\\
\hline \hline
$x^2$&$x \cdot x$&$x=0$\\
$x^2+x$&&\\
$x^2+1$&&\\
$x^2+x+1$&&\\
$x^3$&&\\
$x^3+x^2$&&\\
$x^3+x^2+x$&&\\
$x^3+x^2+1$&&\\
$ x^3+x^2+x+1$&&\\
$ x^3+x$&&\\
$x^3+x+1$&&\\
$x^3+1$&&\\
\end{tabular}
\end{center}

\item Prove or Disprove: For every $n \in \{1,2,3,4,5\}$ there exists a polynomial $p(x)$ of degree $n$  with more than $n$ distinct zeros.\\

\begin{center}
\begin{tabular}{lll}
degree & polynomial&zeros\\
\hline \hline
1&&\\
2&&\\
3&&\\
4&&\\
5&&\\
\end{tabular}
\end{center}

\item Let $I=\langle x^3+2x^2 \rangle$ be an ideal in $\bbQ[x].$\\
	\begin{enumerate}
	\item List 5 distinct elements of $\bbQ[x]$ that are in $I,$ list 5 distinct elements of $\bbQ[x]$ that are \emph{not} in $I,$ and then describe in words or symbols what the ideal $I$ looks like. \\
	
	\item Is $I$ a maximal ideal? Justify your answer.\\
	
	
	\end{enumerate}

\item Suppose $f(x)$ is irreducible in $F[x]$, where $F$ is a field. Prove that for every nonzero polynomial $g(x) \in F[x],$ either $\text{gcd}(f(x),g(x))=1$ or $f(x) \, \big\vert g(x).$

\item 
	\begin{enumerate}
	\item Rewrite Fermat's Little Theorem as stated in your text book.\\
	
	\item Show that Fermat's Little Theorem applies for $n=5,$ by explicitly calculating $a^{4}$ for every $a \in \{1,2,3,4\}$ and checking that each is congruent to $1.$\\
	
	\item Explain what Fermat's Little Theorem implies about factorizations of $x^p-x$ in $\bbZ_p[x]$ where $p$ is a prime. Prove that your assertion is correct.
	\end{enumerate}
	
\item Show that $\alpha=\sqrt{5}+\sqrt{3}i$ is algebraic over $\bbQ[x]$ by explicitly finding $p(x) \in \bbQ[x]$ such that $p(\alpha)=0.$

\end{enumerate}
\end{document}

	
	
	
	
	
	

 

