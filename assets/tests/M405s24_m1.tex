% !TEX TS-program = pdflatexmk
\documentclass[12pt]{article}

% Layout.
\usepackage[top=.75in, bottom=0.75in, left=.75in, right=.75in, headheight=1in, headsep=6pt]{geometry}

% Fonts.
\usepackage{mathptmx}
%\usepackage[scaled=0.86]{helvet}
\renewcommand{\emph}[1]{\textsf{\textbf{#1}}}
\renewcommand{\familydefault}{\sfdefault}
%Syl adendum
\usepackage[colorlinks = true,linkcolor = blue, urlcolor  = blue]{hyperref}
\def\mailto#1{\href{mailto:#1}{#1}}

% Misc packages.
\usepackage{amsmath,amssymb,latexsym,multicol}
\usepackage{graphicx}
\usepackage{array}
\usepackage{xcolor}
\usepackage{multicol}
\usepackage{tabularx,colortbl}
\usepackage{enumitem}
%to make tikz pics work
\usepackage{tikz,pgfplots}

\usepackage[colorlinks=true]{hyperref}

% Paragraph spacing
\parindent 0pt
\parskip 6pt plus 1pt
\def\tableindent{\hskip 0.5 in}
\def\ts{\hskip 1.5 em}

\usepackage{fancyhdr}
\pagestyle{fancy} 
\lhead{\large\sf\textbf{MATH 405 }}
\rhead{\large\sf\textbf{Spring 2024}}
\chead{\large\sf\textbf{Midterm I}}

\newcommand{\localhead}[1]{\par\smallskip\textbf{#1}\nobreak\\}%
\def\heading#1{\localhead{\large\emph{#1}}}
\def\subheading#1{\localhead{\emph{#1}}}

\newenvironment{clist}%
{\bgroup\parskip 0pt\begin{list}{$\bullet$}{\partopsep 4pt\topsep 0pt\itemsep -2pt}}%
{\end{list}\egroup}%

\usetikzlibrary{calc}
\pgfplotsset{my style/.append style={axis x line=middle, axis y line=
middle, xlabel={$x$}, ylabel={$y$}, axis equal }}


\begin{document}
\quad
\vskip 2cm
\strut\vtop{\halign{\emph#\hskip 0.5em\hfil&#\hbox to 2in{\hrulefill}\cr
\emph{\fontsize{18}{22}\selectfont Name:}&\cr
\noalign{\vskip 10pt}}}

\vfill
{\fontsize{18}{22}\selectfont\emph{Rules:}}

You have 1.5 hours to complete this midterm.  

Partial credit will be awarded, but you must show your work.

No notes, books, or other aids are allowed.

Turn off anything that might go beep during the exam.

Good luck!
\vfill
\def\emptybox{\hbox to 2em{\vrule height 16pt depth 8pt width 0pt\hfil}}
\def\tline{\noalign{\hrule}}
\centerline{\vbox{\offinterlineskip
{
\bf\sf\fontsize{18pt}{22pt}\selectfont
\hrule
\halign{
\vrule#&\strut\quad\hfil#\hfil\quad&\vrule#&\quad\hfil#\hfil\quad
&\vrule#&\quad\hfil#\hfil\quad&\vrule#\cr
height 3pt&\omit&&\omit&&\omit&\cr
&Problem&&Possible&&Score&\cr\tline
height 3pt&\omit&&\omit&&\omit&\cr
&1&&10&&\emptybox&\cr\tline
&2&&10&&\emptybox&\cr\tline
&3&&25&&\emptybox&\cr\tline
&4&&25&&\emptybox&\cr\tline
&5&&15&&\emptybox&\cr\tline
&6&&15&&\emptybox&\cr\tline
&Extra Credit&&5&&\emptybox&\cr\tline
&Total&&100&&\emptybox&\cr
}\hrule}}}

\newpage
\begin{enumerate}
\item (10 points) Use the method of induction to prove the statement below.\\
\begin{quote} For all integers $n\geq 1,$ $$1\cdot 2 +2 \cdot 2^2 +3 \cdot 2^3 + \cdots + n \cdot 2^n=(n-1) \cdot 2^{n+1}+2.$$ \end{quote}

\newpage
\item (10 points) Suppose $a,b,$ and $c$ are nonzero integers. Prove that $\text{gcd}(a,bc)=1$ if and only if $\text{gcd}(a,b)=1$ and $\text{gcd}(a,c )=1.$

\vfill 
\newpage
\item (25 points) Give an examples of the following, if they exist. Otherwise briefly explain why such examples do not exist.
	\begin{enumerate}
	\item An infinite nonabelian group.
	\vfill
	\item An abelian group of order $n$ for any integer $n$, $n\geq 1.$
	\vfill
	\item A group $G$  with exactly two subgroups.
	\vfill
	\item An infinite cyclic group.
	\vfill
	\item A nonabeliean group such that every proper subgroup is abelian.
	\vfill
	\end{enumerate}
\newpage
	
\item (25 points) Let $\sigma=(1 \, 2 \, 3 \, 4 \, 5 ),$  $\tau = (2 \, 4 \, 3 \, 6)$, and $\rho=(1 \, 2 \, 3 )(2 \,4) (2 \, 6 \, 4) (4 \, 5)$ be elements of $S_6,$ the symmetric group on 6 letters.
	\begin{enumerate}
	\item Find $\sigma \circ \tau (2)$ and $\tau \circ \sigma (2).$
	\vfill
	\item Determine $| \sigma |,$ the order of $\sigma$ in $S_6.$
	\vfill
	\item Write $\rho$ as a product of disjoint cycles.
	\vfill
	\item Write $\rho$ as a product of transpositions.	
	\vfill
	\item Write $(\sigma \circ \tau )^{-1}$ as a product of disjoint cycles.
	\vfill
	\end{enumerate}
\newpage
\item (15 points)
	\begin{enumerate}
	\item State the definition of a group.
	\vspace{2in}
	\item Let $X$ be the set of bijections from $\mathbb{R}$ to $\mathbb{R}.$ Show that the set $X$ under the operation of function composition is a group. 
	\vfill
	\end{enumerate}
\newpage
\item (15 points) Short Answer
	\begin{enumerate}
	\item Determine the order of the element $3$ in the group $( \mathbb{Z}_{18}, +),$ the integers under addition modulo 18 and find the inverse of 3.
	\vfill
	\item Identify the elements of $U(9)$, determine the order of $4,$ and identify the inverse of 4.
	\vfill
	\item Let $a$ be an element of $G,$ a group. If $a^{12}=e,$ what are the possible orders of $a$?
	\vfill
	\end{enumerate}
\newpage
\item (5 points) Prove that if $G$ is a group such that for every $x,y \in G,$ $xy=x^{-1}y^{-1},$ then $G$ is abelian.
\end{enumerate}
\end{document}