% !TEX TS-program = pdflatexmk
\documentclass[12pt]{article}

% Layout.
\usepackage[top=.75in, bottom=0.75in, left=.75in, right=.75in, headheight=1in, headsep=6pt]{geometry}

% Fonts.
\usepackage{mathptmx}
%\usepackage[scaled=0.86]{helvet}
\renewcommand{\emph}[1]{\textsf{\textbf{#1}}}
\renewcommand{\familydefault}{\sfdefault}
%Syl adendum
\usepackage[colorlinks = true,linkcolor = blue, urlcolor  = blue]{hyperref}
\def\mailto#1{\href{mailto:#1}{#1}}

% Misc packages.
\usepackage{amsmath,amssymb,latexsym,multicol}
\usepackage{graphicx}
\usepackage{array}
\usepackage{xcolor}
\usepackage{multicol}
\usepackage{tabularx,colortbl}
\usepackage{enumitem}
%to make tikz pics work
\usepackage{tikz,pgfplots}

\usepackage[colorlinks=true]{hyperref}

%% Special Math Symbol shortcuts
\newcommand{\bbN}{\mathbb{N}}
\newcommand{\bbZ}{\mathbb{Z}}
\newcommand{\bbR}{\mathbb{R}}
\newcommand{\bbQ}{\mathbb{Q}}

% Paragraph spacing
\parindent 0pt
\parskip 6pt plus 1pt
\def\tableindent{\hskip 0.5 in}
\def\ts{\hskip 1.5 em}

\usepackage{fancyhdr}
\pagestyle{fancy} 
\lhead{\large\sf\textbf{MATH 405 }}
\rhead{\large\sf\textbf{Spring 2024}}
\chead{\large\sf\textbf{Midterm II Solutions}}

\newcommand{\localhead}[1]{\par\smallskip\textbf{#1}\nobreak\\}%
\def\heading#1{\localhead{\large\emph{#1}}}
\def\subheading#1{\localhead{\emph{#1}}}

\newenvironment{clist}%
{\bgroup\parskip 0pt\begin{list}{$\bullet$}{\partopsep 4pt\topsep 0pt\itemsep -2pt}}%
{\end{list}\egroup}%

\usetikzlibrary{calc}
\pgfplotsset{my style/.append style={axis x line=middle, axis y line=
middle, xlabel={$x$}, ylabel={$y$}, axis equal }}


\begin{document}
\begin{enumerate}
\item (10 points) \textcolor{red}{(HW 4 Problem 4)}Let $H$ and $K$ be subgroups of the group $G.$ Recall that $HK=\{hk \, : \, h \in H \text{ and } k \in K \}.$ Prove that {if $G$ is abelian}, then $HK \leq G.$\\
\vfill
\textbf{Answer:} Let $H$ and $K$ be subgroups of the abelian group $G.$ Let $HK=\{hk \, : \, h \in H \text{ and } k \in K \}.$\\
(Show $HK$ is closed.) Let $h_1k_1,h_2k_2 \in HK.$ Since $G$ is abelian, $h_1k_1h_2k_2=h_1h_2k_1k_2=h_3k_3,$ where $h_3 \in H$ and $k_3 \in K.$ Thus, $h_1k_1h_2k_2 \in HK.$\\
(Show $e \in HK.$) Since $H$ and $K$ are subgroups, $e \in H$ and $e \in K.$ Thus, $e=ee \in HK.$\\
(Show $HK$ contains inverses.) Let $hk \in HK$ where $h\in H$ and $k \in K.$ Since $H$ and $K$ are subgroups, $h^{-1} \in H$ and $k^{-1} \in K.$ Thus, $h^{-1}k^{-1}=HK.$ Since $G$ is abelian and using our knowledge of inverses, we know that $h^{-1}k^{-1}=k^{-1}h^{-1}=(hk)^{-1}.$ Thus, $(hk)^{-1} \in S.$\\
Since we have shown that $HK$ is closed, contains the identity and inverses, $HK$ is a subgroup.\\


\item (16 points) Give an examples of the following, if they exist. Otherwise briefly explain why such examples do not exist.
	\begin{enumerate}
	\item a noncyclic group of order 13\\
	
	None exists since all groups of prime order are cyclic.\\
	\vfill
	\item an infinite group $G$ and element $g \in G,$ such that $|g|$ is finite\\ 
	In retrospect, I should have added the word \emph{nontrivial} element $g.$\\
	
	$G=\bbZ_2 \times \bbZ$ and element $(1,0).$\\
	\vfill
	\item two nonisomorphic groups of order 18\\
	
	many examples here: $D_9$ or $\bbZ_{18}$ or $\bbZ_3 \times \bbZ_3 \times \bbZ_2$
	\vfill
	\item a nonabelian group $G$ and a subgroup $H$ of $G$ such that $H \lhd G$\\
	
	many examples here but the easiest is $G=S_n$ and $H=A_n$
	\vfill
	\end{enumerate}
\newpage

%%HW 5 #7
\item (20 points) 
	\begin{enumerate}
	\item Suppose that $H$ is a subgroup of the group $G$. Define the \emph{index} of $H$ in $G.$\\
	
	The index of $H$ in $G$ is the number of left (or right) cosets of $H$.\\
	\vfill
	\item Give an example of a group $G$ and a \emph{nontrivial} subgroup $H$ such that the index of $H$ in $G$ is 50.\\
	
	Let $G=\bbZ_{100}$ and $H=\{1,50\}.$\\
	\vfill
	\item \textcolor{red}{(HW 5 Problem 7)} Let $H$ be a subgroup of the group $G.$ Prove that if $[G : H]=2,$ then for every $a,b \in G\backslash H,$ $ab \in H.$\\
	
	\textbf{Proof:} Let $H$ be a subgroup of the group $G$ of index 2. Let $a,b \in G\backslash H.$ Since $a \not \in H$, it follows that $a^{-1} \not \in H.$ Since $a^{-1},b \not \in H,$ we know $a^{-1}H \not = H$ and $bH \not = H.$ Since $[G:H]=2,$ $a^{-1}H=bH.$ By Lemma 6.3, we know that if $a^{-1}H=bH,$ then $(a^{-1})^{-1}b=ab \in H.$
	\end{enumerate}

%%HW6 #9
\item (15 points) (\textcolor{red}{(HW 6 Problem 9)} Let $G$ be a group and $g \in G.$ Define the function $f_g(x) : G \to G$ by $f_g(x)=gxg^{-1}.$ Prove that $f_g$ is an isomorphism from $G$ to itself.\\

\textbf{Proof:}\\
(Show $f_g$ is 1-1.) Let $x,y \in G$ such that $f_g(x)=f_g(y).$ Then by the definition of $f_g,$ $gxg^{-1}=gyg^{-1}.$ Using left and right cancellation, we obtain $x=y.$\\

(Show $f_g$ is onto.) Let $y \in G.$ Pick $x=g^{-1}yg \in G.$ Then $f_g(x)=f_g(g^{-1}yg)=gg^{-1}ygg^{-1}=y.$\\

(Show $f_g$ respects op.) Let $x,y \in G.$ Observe 

\begin{tabular}{rlr}
$f(xy)$&$=gxyg^{-1}$& by the definition of $G$\\
&$=gxeyg^{-1}$ & by the definition of $e$\\
&$=gxg^{-1}gyg^{-1}$ & since $e=g^{-1}g$\\
&$=f(x)f(y)$& by the definition of $f$.
\end{tabular}


\newpage
%\item (10 point) Let $H$ be a subgroup of the group $G$ and let $a \in G.$ Prove that $aH \leq G$ if and only if $a \in H.$

\vfill

\item (12 points)
	\begin{enumerate}
	\item State Lagrange's Theorem\\
	
	Suppose that $H$ is a subgroup of the group $G.$ Then,\\
	
	\begin{itemize}
	\item $\displaystyle [G:H]=\frac{|G|}{|H|}$ and 
	\item $|H| \, \big \vert \, |G|$
	\end{itemize}
	\vfill
	\item Suppose $K$ is a proper subgroup of $H$ and $H$ is a proper subgroup of $G.$ (So $K<H<G.$) If $|K|=10$ and $|G|=200,$ what are the possible orders of $H$? (Or, what are the possible values for $|H|$?) Explain your reasoning.\\
	
	Since $K \leq H$ and $|K|=10$, Lagrange's Theorem implies that $|H|=10n$ for some integer $n.$ Since $K$ is a \emph{proper} subgroup, $n\geq 2.$ Since $H$ is a subgroup of $G$ and $|G|=200,$ Lagrange's Theorem implies that $|H| \, \big \vert \, |G|$ or, equivalently, $10n \big \vert 200.$ Since $H$ is a \emph{proper} subgroup of $G$, we conclude that $n < 20.$\\
	
	Using the fact that $200=2^35^2=10\cdot2^2\cdot 5,$ we conclude the possible values of $n$ are: 2,4,5, and 10. So, the possible values of $|H|$ are $20,\; 40,\; 50,\;$ and $100.$\\	
	\vfill
	\end{enumerate}
\item (12 points)
	\begin{enumerate}
	\item State the definition of a \emph{normal subgroup.}\\
	
	The subgroup $H$ of $G$ is called \emph{normal} if for every $g \in G,$ $gH=Hg.$
	\vfill
	\item Let $G=GL_2(\bbR)$ and let $H=\{ A \in GL_2(\bbR) \,  : \, \text{det}(A)=2^k \text{ for } k \in \bbZ\}$ be a subgroup of $G.$ Prove that $H \lhd G.$ \\
	\emph{Note:} You do not need to prove that $H$ is a subgroup of $G.$ You only need to prove that $H$ is normal in $G.$\\
	
	\textbf{Proof:} Let $G=GL_2(\bbR)$ and let $H=\{ A \in GL_2(\bbR) \,  : \, \text{det}(A)=2^k \text{ for } k \in \bbZ\}.$ By Theorem 10.3, we know that $H \lhd G$ if and only if $ghg^{-1} \in H$ for every $g \in G$ and every $h \in H.$ 
	
	Let $B \in GL_2(\bbR)$ and $A \in H.$ We need to show that $BAB^{-1} \in H.$ Using properties of the determinant, we know that 
	$$det(BAB^{-1})=det(B)det(A)det(B^{-1})=det(B)det(B^{-1})det(A)=1 \cdot det(A).$$
	But $det(A)=2^k$ for some integer $k$, since $A \in H.$ Thus, $det(BAB^{-1})=2^k$ for some integer $k.$ Thus, $BAB^{-1} \in H.$ Thus, $H \lhd G.$
	\vfill
	\end{enumerate} 

\newpage

\item (15 points) Let $G=\bbZ_{24}$ and $H=\langle 8 \rangle.$
	\begin{enumerate}
	\item In one or two sentences, explain why $H \lhd G.$\\
	
	
	$G$ is abelian so all subgroups of $G$ are normal.\\
	
	\item List the distinct elements of $G/H.$\\
	
	$0+H,\: 1+H,\:2+H,\:3+H,\:	4+H,\:5+H,\:6+H,\:7+H$\\
	
	\item For cosets $10+H$ and $20 +H$, determine $(10+H)+(20+H)$\\
	
	$30+H=6+H$\\
	
	\item Determine the order of the element $10 + H$ in $G/H.$\\
	
	Since $10 + H=2+H$, we can see that $4 \cdot (2+H)=0+H$ and $4$ is the smallest such value. So $|2+H|=4.$\\
	
	\item Can $G/H$ have an element of order 5? If so, find such an element. If not, explain why it is not possible.\\
	
	\textbf{Answer:} No.\\
	\textbf{Reasoning:} Since the order of $G/H$ is 8, the order of every element must divide $8.$
	\end{enumerate}

\end{enumerate}
\emph{Extra Credit:} (5 points) Suppose that $H$ and $K$ are subgroups of the group $G.$ Prove that if there exist elements $a,b \in G$ such that $aH \subseteq bK,$ then $H \subseteq K.$

\textbf{Proof:} Suppose that $H$ and $K$ are subgroups of the group $G.$ Suppose there exist elements $a,b \in G$ such that $aH \subseteq bK.$

Let $h \in H.$ We need to show that $h \in K.$ 

Since $a=ae \in aH$ and $aH \subseteq bK,$ we know there exists some $k \in K$ such that $a=bk$ and, consequently, $a^{-1}=k^{-1}b^{-1}.$  

Since $h \in H$ and $aH \subseteq bK,$ we know there exists some $k' \in K$ such that $ah=bk'.$ Operating on the left by $a^{-1}$, we conclude $$h=a^{-1}bk'=k^{-1}b^{-1}bk'=k^{-1}k'.$$
 Since $K$ is a group, $k^{-1}k'$ is in $K.$ Thus, $h \in K.$
\end{document}