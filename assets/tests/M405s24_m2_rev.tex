% !TEX TS-program = pdflatexmk
\documentclass[12pt]{article}

% Layout.
\usepackage[top=.75in, bottom=0.75in, left=.75in, right=.75in, headheight=1in, headsep=6pt]{geometry}

% Fonts.
\usepackage{mathptmx}
\usepackage[scaled=0.86]{helvet}
\renewcommand{\emph}[1]{\textsf{\textbf{#1}}}

% Misc packages.
\usepackage{amsmath,amssymb,latexsym}
\usepackage{graphicx,tikz}
\usepackage{array}
\usepackage{xcolor}
\usepackage{multicol}
\usepackage{tabularx,colortbl}
\usepackage{enumitem}
%to make tikz pics work
\usepackage{tikz,pgfplots}
\usetikzlibrary{arrows}
\newcommand{\midarrow}{\tikz \draw[-triangle 90] (0,0) -- +(.1,0);}

\usepackage[colorlinks=true]{hyperref}

% Paragraph spacing
\parindent 0pt
\parskip 6pt plus 1pt
\def\tableindent{\hskip 0.5 in}
\def\ts{\hskip 1.5 em}

\usepackage{fancyhdr}
\pagestyle{fancy} 
\lhead{\large\sf\textbf{MATH 405}}
\rhead{\large\sf\textbf{Spring 2024}}
\chead{\large\sf\textbf{Midterm 2 Review }}

\newcommand{\localhead}[1]{\par\smallskip\textbf{#1}\nobreak\\}%
\def\heading#1{\localhead{\large\emph{#1}}}
\def\subheading#1{\localhead{\emph{#1}}}

%% Special Math Symbol shortcuts
\newcommand{\bbN}{\mathbb{N}}
\newcommand{\bbZ}{\mathbb{Z}}
\newcommand{\bbR}{\mathbb{R}}
\newcommand{\bbQ}{\mathbb{Q}}

\newcommand{\rad}{\text{rad}}
\newcommand{\diam}{\text{diam}}
\newcommand{\divs}{\, \big | \,}

\usetikzlibrary{calc,arrows.meta}
\usetikzlibrary{arrows}
\newcommand{\marrow}{\tikz \draw[-triangle 90] (0,0) -- +(.1,0);}


\begin{document}
\textbf{Logistics:} Midterm I will be Thursday March from 2:00-3:30 for in-person students. For remote students, it will be on either Thursday February 8 or Friday February 9. No notes, books or other aids.\\

\textbf{Reminders:}
\begin{enumerate}
	\item Know the \textbf{formal definition}. Intuitive definitions are important for understanding but proofs require the use of the formal definition. If you are unsure of the formal definition, ask; don't guess.
	\item All proofs should be formal and adhere to the same expectations as your written homework including the use of complete sentences, a clear beginning and conclusion, and appropriate use of symbols.
	\item Unless explicitly stated otherwise, all answers require a rigorous explanation.
	\item The emphasis will be on Chapters 5, 6, 9 and 10.
\end{enumerate}

\textbf{Suggestions:}
\begin{enumerate}
	\item Read over your commented-on homework. If a problem has a circle around the number, you would not have gotten full-credit for that problem. Do you understand why something is marked as incorrect or missing? How can you not make that mistake again?
	\item Read over my solutions to the homework. What details in my solutions are absent from yours? Did you and I prove things the same way? If mine was different, does it have any advantages? What are the things from my solutions you want to make sure to include in the future?
	\item Make a list of all the examples of groups we have discussed thus far. Which are cyclic or not? Abelian or not? Finite or infinite? Which ones are isomorphic to each other?
	\item Look at old Midterms.
	\item Look at other problems from the text.
\end{enumerate}

\textbf{New Topics}\\

\noindent \textbf{More Chapter 5}\\

\noindent \textbf{Definitions:} the alternating group, the dihedral group

\noindent \textbf{Notation:} $A_n$, $D_n$

\noindent \textbf{Results:}
\begin{itemize}
	\item (Thm 5.16, Prop 5.17) $A_n \leq S_n$ and $|A_n| = \frac{1}{2}|S_n|$
	\item (Thm 5.20, 5.21) $D_n \leq S_n,$ $|D_n|=2n$ and $D_n$ can be generated by a rotation by $360/n$ and a single reflection.
\end{itemize}

\noindent \textbf{Chapter 6}\\

\noindent \textbf{Definitions:} left coset, right coset, coset representative, the index of $H$ in $G,$ Lagrange's Theorem

\noindent \textbf{Notation:} $gH$ and $Hg,$ $[G:H],$

\noindent \textbf{Results:}
\begin{itemize}
	\item (Lemma 6.3) $H\leq G$ and $g_1,g_2 \in G$. TFAE
		\begin{itemize}
		\item $g_1H=g_2H$
		\item $Hg_1^{-1}=Hg_2^{-1}$
		\item $g_1H \subseteq g_2H$
		\item $g_2 \in g_1H$
		\item $g_1^{-1}g_2 \in H$
		\end{itemize}
	\item (Thm 6.4) Left (right) cosets of $H$ in $G$ partition $G.$
	\item (Thm 6.8) The number of left cosets of $H$ in $G$ is the same as the number of right cosets of $H$ in $G$. Consequently the notion of the \emph{index} of $H$ is $G$ is well-defined.
	\item (Prop 6.9) For $H \leq G$ and $g \in G,$ the map $\phi:G \to G$ defined as $\phi(x)=gx$ is a bijection (or, equivalently, $\phi$ defines a permutation of $G$). Consequently, for every $H \leq G$ and $g \in G,$ $|H|=|gH|.$
	\item (Thm 6.10, Lagrange's Theorem) $H \leq G,$ a finite group. Then
		\begin{itemize}
		\item $\frac{|G|}{|H|}=[G:H]$, and
		\item $|H| \, \Big| \, |G|.$
		\end{itemize}
	\item (Cor 6.11, 6.12, 6.13 Consequences of LT) 
		\begin{itemize}
		\item $\forall g \in G$ finite, $|g| \, \Big| \, |G|.$
		\item $\forall G$ such that $|G|=p,$ a prime, $G$ is cyclic.
		\item If $K \leq H \leq G$ finite, then $[G:K]=[G:H][H:K].$
		\end{itemize}
	\item (Prop 6.15 that the converse of LT is false) While $6 \, \Big| \, A_4,$ there does not exist a subgroup of $A_4$ of order 6.
\end{itemize}

\noindent \textbf{Chapter 9}\\

\noindent \textbf{Definitions:} group $G$ is isomorphic to group $H,$ external direct product, internal direct product

\noindent \textbf{Notation:} $G \times H$, $G=HK$

\noindent \textbf{Results:}
	\begin{itemize}
	\item (Thm 9.6) $\phi: G \to H$ group isomorphism. TFAE
		\begin{itemize}
		\item $\phi^{-1}: H \to G$ group isomorphism
		\item $|G|=|H|$
		\item If one is abelian, the other is abelian
		\item If one is cyclic, the other is cyclic
		\item If one has a subgroup of order $n$, then the other has a subgroup of order $n.$
		\end{itemize}
	\item (Thm 9.7, 9.7) All infinite cyclic groups are isomorphic to $\bbZ$ and all cyclic groups of order $n$ are isomorphic to $\bbZ_n.$
	\item (Cor 9.9) Every group of prime order $p$ is isomorphic to $\bbZ_p.$
	\item (Thm 9.12 Cayley's Theorem) Every group is isomorphic to a group of permutations.
	\item (Thm 9.17) Let $g \in G$ and $h \in H$ such that $|g|=r$ and $|h|=s$, then the element $(g,h)$ in the group $G \times H$ has order $lcm(r,s).$
	\item (Thm 9.27) If $G$ is the internal direct product of $H$ and $K$, then $G$ is isomorphic to $H \times K.$
	\end{itemize}

\noindent \textbf{Chapter 10}\\

\noindent \textbf{Definitions:} normal subgroup, factor group or quotient group, simple group

\noindent \textbf{Notation:} $N \lhd G$, $G/N$

\noindent \textbf{Results:}
	\begin{itemize}
	\item (Thm 10.3) $N \leq G$. TFAE
		\begin{itemize}
		\item $N \lhd G$
		\item $\forall g \in G,$ $gNg^{-1} \subseteq N$
		\item $\forall g \in G,$ $gNg^{-1} = N$
		\end{itemize}
	\item (Thm 10.4) If $N \lhd G,$ then the set of cosets of $N$ in $G$ form a group of order $[G:N]$ with group operation $(aN)(bN)=abN.$
	\item (Thm 10.11) $A_n$ is simple for $n \geq 5.$
	\end{itemize}


\textbf{Old Topics that are implicit in new topics:}\\

\noindent \textbf{Chapter 1}\\

\textbf{Definitions:} Cartesian product, functions, image, one-to-one/injective, onto/surjective, bijective\\

\noindent \textbf{Chapter 3}\\

\textbf{Definitions:} binary operation, associativity, identity, inverse, commutativity, group, order of a group, group of symmetries of an object, addition and multiplication modulo $n$, group of units, general linear group, special linear group, subgroup, proper subgroup, trivial subgroup

\textbf{Notation:} $(\bbZ,+)$ and with $\bbR,\bbQ,$  $(\bbZ_n,+)$, $(U(n), \cdot )$, $GL_n(\bbR)$, $|G|$, $SL_2(\bbR)$

\textbf{Results:}
\begin{itemize}
	\item If $G$ is a group, then
		\begin{itemize}
		\item (Prop 3.17) the identity is unique.
		\item (Prop 3.18) $\forall a \in G,$ $a^{-1}$ is unique.
		\item (Props 3.19 and 3.20) $\forall a,b \in G,$ $(ab)^{-1}=b^{-1}a^{-1}$ and $(a^{-1})^{-1}=a.$
		\item (Prop 3.21) $\forall a,b \in G,$ equation $ax=b$ and $xa=b$ have unique solutions.
		\item (Prop 3.22) $\forall a,b \in G,$ both equations $ab=ac$ and $ba=ca$ imply $b=c.$
		\end{itemize}
	\item (Thm 3.23) We can use the usual laws of exponents when manipulating repeated group operations on a single element. Specifically, if $g,h \in G$ a group and $m,n \in \bbZ,$ then
		\begin{itemize}
		\item $g^mg^n=g^{m+n}$
		\item $(g^m)^n=g^{mn}$
		\item $(gh)^n=((gh)^{-1})^{-n}=(h^{-1}g^{-1})^{-n}$
		\end{itemize}
	\item $H$ is a subgroup of $G$ if and only if
		\begin{itemize}
		\item (Prop 3.30) (i) $e \in H,$ (ii) $h_1h_2 \in H$ for every $h_1,h_2 \in H,$ and (iii) $h^{-1} \in H$ for every $h \in H.$
		\item (Prop 3.31) (i) $H \not= \emptyset$ and (ii)$gh^{-1} \in H$ for every $g,h \in H.$
		\end{itemize}
\end{itemize}

\noindent \textbf{Chapter 4}\\

\textbf{Definitions:} Cyclic group, cyclic subgroup, generator of a group, cyclic subgroup generated by $a$, order of an element of a group,

\textbf{Notation:} $\langle a \rangle$, $|b|,$ $n \bbZ$

\textbf{Results:}
\begin{itemize}
	\item (Thm 4.9) Cyclic groups are abelian.
	\item (Thm 4.10) Every subgroup of a cyclic group is cyclic.
	\item (Prop 4.12) If $G=\langle a \rangle$ of order $n$, then $a^k=e$ if and only of $n \divs k.$
	\item (Thm 4.13)  If $G=\langle a \rangle$ of order $n$ and $b=a^\ell \in G$, then $|b|=\frac{n}{d}$ where $d=gcd(n,\ell).$
	\item (Cor 4.11 of Prop 4.12) The subgroups of $(\bbZ,+)$ are $\langle 1 \rangle=\bbZ, \, \langle 2 \rangle=2\bbZ, \, \langle 3 \rangle=3\bbZ, \cdots.$
	\item  (Cor 4.14 of Thm 4.13) Suppose $1 \leq r < n.$ Then,
	$\bbZ_n=\langle r \rangle$ if and only if $gcd(n,r)=1.$
\end{itemize}



\noindent \textbf{Chapter 5 Section 1}\\

\textbf{Definitions} permutation, the symmetric group on $n$ letters, a permutation group, disjoint cycles, transposition, even/odd permutation, length of a permutation.

\textbf{Notation:} $S_n$, permutation notation including disjoint cycle notation and transposition representation

\textbf{Results}
\begin{itemize}
	\item (Thm 5.1)The set of all permutations of the set $X$ under function composition is a group.
	\item (Prop 5.8) Disjoint cycles permute.
	\item (Thm 5.9) Every permutation can be written as a product of disjoint cycles.
	\item (Prop 5.12) Any permutation of a finite set can be written as a product of transpositions, provided the set has at least two elements.
	\item (Thm 5.15) For every permutation $\sigma,$ the parity (even or odd) of the number of transpositions in any transposition representation of $\sigma$ is fixed. (i.e. always even or always odd).
\end{itemize}

\noindent \textbf{Things you will be asked}
\begin{itemize}
	\item An example question
	\item To formally state Lagrange's Theorem
	\item To reproduce two of the homework problems that involved proofs.
\end{itemize}

\end{document}