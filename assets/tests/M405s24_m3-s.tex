% !TEX TS-program = pdflatexmk
\documentclass[12pt]{article}

% Layout.
\usepackage[top=.75in, bottom=0.75in, left=.75in, right=.75in, headheight=1in, headsep=6pt]{geometry}

% Fonts.
\usepackage{mathptmx}
%\usepackage[scaled=0.86]{helvet}
\renewcommand{\emph}[1]{\textsf{\textbf{#1}}}
\renewcommand{\familydefault}{\sfdefault}
%Syl adendum
\usepackage[colorlinks = true,linkcolor = blue, urlcolor  = blue]{hyperref}
\def\mailto#1{\href{mailto:#1}{#1}}

% Misc packages.
\usepackage{amsmath,amssymb,latexsym,multicol}
\usepackage{graphicx}
\usepackage{array}
\usepackage{xcolor}
\usepackage{multicol}
\usepackage{tabularx,colortbl}
\usepackage{enumitem}
%to make tikz pics work
\usepackage{tikz,pgfplots}

\usepackage[colorlinks=true]{hyperref}

%% Special Math Symbol shortcuts
\newcommand{\bbN}{\mathbb{N}}
\newcommand{\bbZ}{\mathbb{Z}}
\newcommand{\bbR}{\mathbb{R}}
\newcommand{\bbQ}{\mathbb{Q}}

% Paragraph spacing
\parindent 0pt
\parskip 6pt plus 1pt
\def\tableindent{\hskip 0.5 in}
\def\ts{\hskip 1.5 em}

\usepackage{fancyhdr}
\pagestyle{fancy} 
\lhead{\large\sf\textbf{MATH 405 }}
\rhead{\large\sf\textbf{Spring 2024}}
\chead{\large\sf\textbf{Midterm III}}

\newcommand{\localhead}[1]{\par\smallskip\textbf{#1}\nobreak\\}%
\def\heading#1{\localhead{\large\emph{#1}}}
\def\subheading#1{\localhead{\emph{#1}}}

\newenvironment{clist}%
{\bgroup\parskip 0pt\begin{list}{$\bullet$}{\partopsep 4pt\topsep 0pt\itemsep -2pt}}%
{\end{list}\egroup}%

\usetikzlibrary{calc}
\pgfplotsset{my style/.append style={axis x line=middle, axis y line=
middle, xlabel={$x$}, ylabel={$y$}, axis equal }}


\begin{document}
\quad
Solutions \\

\begin{enumerate}
\item Let $G$ and $H$ be groups and let $\phi: G \to H$ be a group homomorphism.
	\begin{enumerate}
	\item (2 pts) State the definition of a \emph{group homomorphism}.\\
	
	A function $\phi: G \to H$ is a group homomorphism if $\forall a,b \in G,$
	 $\phi(ab)=\phi (a) \phi(b).$ (or, if you prefer, $\phi(a+b)=\phi (a) + \phi(b).$ \\
	\vfill
	\item (2 pts) State the definition of the \emph{kernel of $\phi,$} $\text{ker } \phi.$ \\
	
	
	Given a group homomorphism $\phi: G \to H$, the \emph{kernel of $\phi,$} $\text{ker } \phi,$ is $\phi^{-1}(0_H)$ or the inverse image of the identity in $H$ or the set of elements in $G$ whose image is the identity in $H.$\\
	
	\item (12 pts) Prove $\text{ker }\phi$ is a normal subgroup of $G.$ (Note that you must show $\text{ker }\phi$ is a subgroup of $G$ \emph{and} that it is normal.)\\
	
	\textbf{Proof:} (\textbf{$\text{ker }\phi$ is a subgroup of $G.$})\\
	We know that all group  homomorphisms send the identity in the domain to the identity in the range. So $e_G \in \text{ker }\phi$ which implies  $\text{ker }\phi \not = \emptyset.$\\
	Let $a,b \in \text{ker }\phi.$ Observe \\
	\begin{tabular}{rlr}
	$\phi(ab^{-1})$&$=\phi(a)\phi(b^{-1})$& b/c $\phi$ respects the group operation\\
	&$=\phi(a)( \, \phi(b)\,)^{-1}$& by Prop 11.4\\
	&$=e_H\cdot (e_H)^{-1}$& b/c $a,b \in \text{ker}\phi$\\
	&$=e_H$& b/c $e_H$ is the identity.\\
	\end{tabular}
	
	Thus, we have shown that $ab^{-1} \in \text{ker }\phi.$ Thus, by Proposition 3.31, the kernel of $\phi$ is a subgroup of $G.$
	
	(\textbf{$\text{ker }\phi$ is normal $G.$})\\
	 By Theorem 10.3, it is sufficient to demonstrate that $gag^{-1} \in \text{ker }\phi,$ for every $g \in G$ and $a \in \text{ker }\phi.$Observe \\
	\begin{tabular}{rlr}
	$\phi(gag^{-1})$&$=\phi(g)\phi(a)\phi(g^{-1})$& b/c $\phi$ respects the group operation\\
	&$=\phi(g)e_H\phi(g^{-1})$& b/c $a \in \text{ker}\phi$ \\
	&$=\phi(g)\phi(g)^{-1}$&by Prop 11.4\\
	&$=e_H$.&\\
	\end{tabular}
	
	Thus, we have shown that $gag^{-1} \in \text{ker }\phi.$\\
	
		\end{enumerate}
\item (20 points) Give an examples of the following, if they exist. Otherwise briefly explain why such examples do not exist.
	\begin{enumerate}
	\item A commutative ring with unity that is not an integral domain.\\
	
	$\bbZ_6$ (Note any $\bbZ_n$ where $n$ is composite would suffice.)
	\vfill
	\item A ring that is an integral domain but is not a field.\\
	
	$\bbZ$ or $\bbR[x]$ \\
	\vfill
	\item A ring $R$ and a nontrivial subring $I$ such that $I$ is an ideal of $R$\\
	
	$R=\bbZ$ and $I=6\bbZ$\\
	
	\vfill
	\item A ring $R$ and a nontrivial subring $S$ such that $S$ is \emph{not} an ideal of $R$\\
	
	$R=\bbZ[x]$ and $S=\bbZ$ or $R=\bbR$ and $S=\bbZ$\\
	\vfill
	\item A ring $R$ and an ideal $I$ that is prime.\\
	
	$R=\bbZ$ and $I=2\bbZ$\\
	\vfill
	\end{enumerate}
%%hwk # 9 + First Isom Theorem
\item (16 points) 
	\begin{enumerate}
	\item (4 pts) State the First Isomorphism Theorem (for groups)
	Let $f: G \to H$ be a group homomorphism with kernel $K.$ Let $g:G \to G/K$ be the canonical homomorphism. Then there is a unique isomorphism $h: G/K \to f(G)$ such that $f=h \circ g.$\\
	\item (12 pts) Let $\psi: G \to H$ be a group homomorphism. Prove that $\psi$ is one-to-one if and only if $\psi^{-1}(e_H)=\{e_G\}.$\\	
	
	\textbf{Proof:} ($\Longrightarrow$:) Suppose that $\psi$ is one-to-one. Since $\psi$ is a homomorphism, $\psi(e_G)=e_H.$ Since $\psi$ is one-to-one, $\psi$ can map no other element of $G$ to $e_H.$ Thus, $\phi^{-1}(e_H)=\{e_G\}.$\\
	
	($\Longleftarrow$:) Suppose $\psi^{-1}(e_H)=\{e_G\}.$ Thus, by the definition of kernel, $\text{ker} \psi = \{e_G\}.$ Since $\text{ker} \psi = \{e_G\},$ it follows that $G \cong G/(\text{ker} \psi ).$ The First Isomorphism Theorem states that $G/(\text{ker} \psi ) \cong \psi(G).$ Thus, $G \cong \psi(G).$ Thus, $\psi$ must be one-to-one.\\
	\end{enumerate}
\item (12 points) Prove that if $R$ is a field, the only ideals of $R$ are $\{0\}$ and $R$ itself.\\

\textbf{Proof:} Let $R$ be a field and let $I$ be an ideal in $R$ such that $I \not = \{0\}.$ Since $I \not =\{0\},$ it follows that there exists some $r \in R \backslash \{0\}$ such that $r \in I.$ Since $R$ is a field and $r \not=0,$ there exists a multiplicative inverse, $r^{-1}$, in $R.$ Since $I$ is an ideal, $r^{-1}r=1 \in I.$ Since $1 \in I,$ for every $a \in R,$ $a=a\cdot 1 \in I.$ Thus, $I=R.$
\vfill

%
\item (12 points) Let $R$ be a ring and let $a \in R.$ Prove that the set $S=\{r \in R \, : \, ra=0\}$ is a subring of $R.$ Note that you should not assume $R$ is commutative.\\

\textbf{Proof}: (Note that I am using Prop 16.10)\\
(Show $S \not = \emptyset$.) We know that $0 \cdot a =0.$ Thus, $0 \in S.$\\
(Show $S$ is closed under multiplication.) Let $x,y \in S.$ Observe\\
\begin{center}
\begin{tabular}{rlr}
$(xy)a$&$=x(ya)$&b/c mult is associative in $R$\\
&$=x\cdot 0$& b/c $y \in S$\\
&$=0$.&\\
\end{tabular} 
\end{center}
Thus $xy \in S.$

(Show $x-y \in S,$ $\forall x,y \in S.$) Let $x,y \in S.$ Observe that 
$$0=0\cdot a = (y+(-y))a=ya+(-y)a=0+(-y)a=(-y)a.$$
Thus, 
$$(x-y)a=xa+(-y)a=0+0=0.$$
Thus, $x-y \in S.$

\vfill

\item (24 points) 
\begin{enumerate}
	\item List all nonisomorphic abelian groups of order 24.\\
	
	\textbf{answer:} $\bbZ_8 \times \bbZ_3,\; \bbZ_2 \times\bbZ_4 \times \bbZ_3,\; \bbZ_2 \times\bbZ_2 \times\bbZ_2 \times \bbZ_3$
	\vfill
	\item Let $\mathbb{R}$ be the ring of real numbers under the usual operations of addition and multiplication. Explain why the function $f:\mathbb{R} \to \mathbb{R}$ defined as $f(x)=2x+1$ is not ring homomorphism.\\
	
	\textbf{answer:} Observe that for real numbers $1$ and $2$, $f(1+2)=f(3)=2\cdot 3+1=7,$ but $f(1)+f(2)=2\cdot 1+1+2\cdot 2+1=8.$ Thus $f$ does not respect addition.\\
	\vfill
	\item Find all group homomorphisms from $\mathbb{Z}_{16}$ to $\mathbb{Z}_{18}.$ Your answer(s) must be stated as functions.\\
	
	\textbf{answer:}  Since $gcd(16,18)=2,$ we know there are two homomorphisms because there are only two divisors of 2, namely 1 and 2.  So,\\
	option 1: $f(x)=0$ (always a homomorphism)\\
	and\\
	option 2: $f(x)=9x$ (b/c $2$ is the only number that divides the orders of both groups, the image of $f$ must have order 2)\\
	\vfill
	\item Give a maximal ideal in the ring $\mathbb{Z}_{20}$\\
	\textbf{answer:} $\langle 2 \rangle$ or $\langle 5 \rangle$ 
	
	\vfill
\end{enumerate}
\end{enumerate}
\emph{Extra Credit:} (5 points) Prove that every finite integral domain is a field.\\

\textbf{Proof:} Let $R$ be a finite integral domain. We must show that for every $r \in R\backslash \{0\}$ there exists an element $r^{-1} \in R$ such that $rr^{-1}=1.$ Observe that $1$ is its own inverse.\\
So let $r \in R\backslash\{0,1\}$ and consider the set $S=\{r^n\, : \, n \in \bbZ^+\}.$  Observe that while $\bbZ^+$ is infinite, the set $S$ must be finite since $S \subset R$ and $R$ is finite. Thus, there exists some $m,n \in \bbZ^+$ such that $m<n$ and $r^m=r^n.$ \\
Since $R$ is an integral domain, the cancellation law applies. Thus, we can conclude $1=r^{n-m}.$ Since $r \not =1,$ we know $n-m\geq 2.$ Thus, we see that $1=r^{n-m}=r\cdot r^{n-m-1}$ where $n-m-1\geq 1.$ So it follows that $r^{n-m-1}$ is the multiplicative inverse of $r$.
\vfill

\end{document}

